%
% DSP Report Template
%
% (c) 2005 EMT DSP Group
%
\documentclass[12pt,a4paper]{article}
%\usepackage{dsp_report}

\usepackage{parskip}

%
%\DSPTitle{Odometry for Wheeled Mobile Robots}
%{Michael Maier}

\usepackage{graphicx}

\usepackage{subfigure}
\usepackage{xspace}                   % space nach makro
\usepackage[left]{eurosym}            % Euro symbol
\usepackage{hyperref}

\newcommand{\mbnote}[1]{\textcolor{Gray}{\textbf{\noindent M. Brandner NOTE: #1}}}
\newcommand{\mmnote}[1]{\textcolor{Gray}{\textbf{\noindent M. Maier NOTE: #1}}}
\newcommand{\MH}{\emph{``Mostly Harmless''} RoboCup Team\xspace}
\newcommand{\MSL}{Middle Size League\xspace}
\newcommand{\robocup}{\emph{RoboCup}\xspace}

\begin{document}

%% Based on a TeXnicCenter-Template by Tino Weinkauf.
%%%%%%%%%%%%%%%%%%%%%%%%%%%%%%%%%%%%%%%%%%%%%%%%%%%%%%%%%%%%%

%%%%%%%%%%%%%%%%%%%%%%%%%%%%%%%%%%%%%%%%%%%%%%%%%%%%%%%%%%%%%
%% Deckblatt
%%%%%%%%%%%%%%%%%%%%%%%%%%%%%%%%%%%%%%%%%%%%%%%%%%%%%%%%%%%%%
%%
%% ATTENTION: You need a main file to use this one here.
%%            Use the command "\input{filename}" in your
%%            main file to include this file.
%%
\begin{titlepage}

  \begin{center}
    \begin{minipage}[htb]{18cm}
      \hspace*{-2.6cm}
      \includegraphics[width=3.3cm]{./figures/logos/EMT.jpg}
      \begin{tabular}{p{10cm}}\centering{
      \Large Institute of Electrical Measurement and Measurement Signal Processing\\ Graz University of Technology
      ~\\
      ~\\}
      \end{tabular}
      \includegraphics[width=3.3cm]{./figures/logos/TUG.jpg}
    \end{minipage}

    \Large {Bachelor's Thesis\\} %
    \vspace*{1cm} \huge{\textbf{Odometry for Wheeled Mobile Robots}\\}
    %
    \vspace*{1.0cm} 
    %\Huge{\textbf{#1}\\}\vspace*{2.5cm} \vfill
    \Large{Michael Maier\\} \vspace*{1cm}
    %
    \Large{Supervisor: Dipl.-Ing. Dr. Markus Brandner\\} \vspace*{0.5cm}%


    \begin{minipage}[htb]{18cm}
      \hspace*{-0.4cm}
      \includegraphics[width=3cm]{./figures/logos/MH.jpg}
      \begin{tabular}{p{10cm}}\centering{
        \Large Mostly Harmless RoboCup Team \\Institute for Software Technology, \\Graz University of Technology\\
        \texttt{team@robocup.tugraz.at}\\ \texttt{http://www.robocup.tugraz.at}
        ~\\
        ~\\}
      \end{tabular}
    \end{minipage}

    \Large{Graz, \today}

  \end{center}
\end{titlepage}


\tableofcontents
\clearpage
\pagestyle{plain}


% seitenanzahl dazuschreiben

\begin{abstract}
Abstract

% 1p
\end{abstract}

\clearpage

\section{Introduction}


\subsection{RoboCup}

The RoboCup is an international research and education initiative. 
It's goal is ``By the year 2050, develop a team of fully autonomous robots that can win against the current human soccer world champion team"~\cite{robocup.org}.
It encourages research in the field of robotics, e.g. machine vision, machine learning, autonomous systems.

%|| to Apollo Program.

Every year, a world championship is held accomplished by a conference in a different city around the globe.
Also, various local competitions and conferences are held by local groups.

The RoboCup is divided into three parts: RoboCup Soccer, Rescue and @Home.
Every division is divided into several leagues targetet at different challenges.

%1/2 page

\subsection{\MSL}

In RoboCup Soccer, the \MSL is the most challenging.
The robots have to be fully autonomous.
All sensors, actuators and computation are on board, no external input is allowed.
A Team consists of at most 6~robots with a size of max. $50\times50\times80$\,cm$^3$.

The game is played on a field of $12\times18$\,m$^2$ and lasts $2\times15$~minutes.
The \MSL is the only league, where an official FIFA ball is used.
Objects are distinguished by color, the game field is green with white lines, robots and referees are black and the ball is red.

The rules are official FIFA rules~\cite{msl-rules} with slight adaptions for robotic players.
The game rules are tightened every year to keep up with the technological progress. 
E.g. the field size has grown from $6\times8$\,m$^2$ to $12\times18$\,m$^2$ since introduction of the \MSL.

More than 20 teams from all over the world participate in the \MSL, almost all with academical background.

% 1/2 p

\subsection{\MH}

The \MH participates in the \MSL. 
It's name is a reference to the ``Hitchhiker's Guide to the Galaxy'' by Douglas Adams~\cite{h2g2}.
It was founded 2003 at Graz University of Technology at the Institute for Software Technology. 
It consists solely of students and acts as a platform for master's, bachelor's thesis and seminar projects.
Currently, more than 30~students are working on the robots in their free time.

It regularily participates at european championships as well as World Championships.
RoboCup 2009 in Graz was the first time the team advanced a round in a World Championship. 
Also, the third place in the technical challenge was won.



% 1/4 p

\subsection{Current Robot Platform: Krikkit}

The ``Krikkit'' robot platform was built in 2006 and is now on the end of its life cycle.


1-2 Images
  
  Specs

    Module
      
    Sensoren
    
    Fahrwerksdynamik

2p

\subsection{In Development: Krikkit3G}

    modular concept
    ARM architecture for every electronics

Image: New rendering

1p

\subsection{Motivation for this Work}
    
Motivation and Goal

1/2p

chaptersum 5p

\section{Odometry}
  Definition
  Verwendungsmoeglichkeit
    Lokalisierung
      higher framerate than cam
      backup-system for cam
    Motion control
      anti-slip regulation

1p

\subsection{Odometry Measurement}

\subsubsection{Current Implementations used}

      wheel encoder
        probleme
      2. satz raeder
        probleme

1p

\subsubsection{Possible Implementations}

      radar
1/4
      accelerometer
        2G spitzenvibration -> no
1/4
      Ultrasonic Doppler Velocimetry
1/4
      Microwave Doppler Velocimetry
1/4
      Laser interferometry
1/2
      Optical Spatial Frequency Filtration
1/4
        Daniel's approach
1/4
        laser speckle applications
        optical mice
1/2

3p

sum 4p

\subsection{Requirements for a new type of Odometry Sensor on the Krikkit/3G platform}

1p


\subsection{Decision for a Type of Sensor}
  entscheidung fuer einen Sensortyp
    - why/why not...
    - ...
    - ...

1p


chaptersum 8p

\section{Designing the Sensor for Evaluation}

block diagram
  ADNS + optic
  XC
  PC

1p

\subsection{The ADNS 6010}

specs

workflow diagram (from datasheet)

1p

\subsubsection{Custom PCB}
ADNS-testplatine
  requirements

  circuit diagram

  PCB drawing

2p

sum 4p
  
\subsection{Optical System}
  requirements

1/2p

\subsubsection{Original Mouse Setup}

1p

\subsubsection{2-Lens Setup}

1p

\subsubsection{Telecentric Setup}

1p

sum 4p
\subsection{The XC164 Microcontroller}

Specs etc

XC-164 eval board 
  Image from datasheet

1/2p

\subsubsection{Wiring of the Evaluation Board}

  block diagram

  circuit

1p

\subsubsection{uC - Software}

  XC-Code
    requirements
    firmware catch

    image streaming

    motion data streaming
      + reference sensor

4p

sum 6p


\subsection{Data Aquisition Software}

\subsubsection{PyQT GUI}
1p

\subsubsection{Octave Analysis Scripts}
1/2p


sum 2p

chaptersum 17p

\section{Experimental Setup}

- requirements

1p

\subsection{Mechanical Design of the Testbed}

block diagram

1p

\subsection{Optical System}

micro-bench system

Sketch

1/2p

\subsubsection{Sensor mounting \& Calibration}

Sketch with adjusting screws in all dimensions

1/2p

calibration process
  laser

1p

\subsubsection{Lens 2 \& Aperture Mounting}

manufacturing drawing

1p

\subsubsection{Illumination}

  Why here?
  types of LEDs and Lasers used
 
  Illumination PCB

  circuit

  PCB print

2p

\subsection{Motor \& Disk}

1/2p

\subsection{Reference Speed Sensor}
      Picture of sensor from Datasheet
      electrical setup used

1p

\subsection{Microcontroller \& Host PC}      

1/2p


chaptersum 9p

\section{Experimental results}

\subsection{Illumination tests}

\begin{itemize}
  \item Laser
  \item Studio Spotlight
  \item ultrabright white LEDs
  \item IR-LEDs
\end{itemize}

\subsection{Non-telecentric tests}

\subsection{Telecentric tests}

\subsubsection{Carpet 1}

\subsubsection{Carpet 2}

\subsubsection{Carpet 3}

\subsubsection{Game Field Lines}

\subsection{Interpretation}

  plots
  ergebnisse

groesser 5p

\section{Outlook - Next Steps}

  evaluation of a combination of IR-LED illumination and Laser illumination

  newer sensor ADNS-90xx

  mechanical limitations
    size constraints
    dirt cleaning...
  electrical requirements
    new architecture: ARM
    also include illumination control in uC
      use of the shuttered illumination for higher power 
      use of both LEDs and Lasers for white lines
  software algorithms: compute 3 dimension out of 2/4/6

ca 3-5p


summe gesamt 49 Seiten (ohne abstract, inhaltsv, bibl.)


%--------------------------------------------------------%
% Bibliography
%--------------------------------------------------------%
\addcontentsline{toc}{section}{Bibliography}
\label{Bibliography}
\bibliographystyle{plain}
\bibliography{literature}
%
\end{document}
